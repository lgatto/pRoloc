\section*{Foreword}

\Biocpkg{MSnbase} and \Biocpkg{pRoloc} are under active developed;
current functionality is evolving and new features will be added.
This software is free and open-source software.  If you use it, please
support the project by citing it in publications:

\begin{quote}
  Gatto L. and Lilley K.S. \emph{\Biocpkg{MSnbase} - an \R/Bioconductor
    package for isobaric tagged mass spectrometry data visualization,
    processing and quantitation.} Bioinformatics 28, 288-289 (2011).
\end{quote}

\begin{quote}
  Gatto L, Breckels LM, Wieczorek S, Burger T, Lilley KS.
  \textit{Mass-spectrometry-based spatial proteomics data analysis
    using \Biocpkg{pRoloc} and \Biocexptpkg{pRolocdata}.}
  Bioinformatics. 2014 Feb 5.
\end{quote}

If you are using the \Rfunction{phenoDisco} function, please also cite

\begin{quote}
  Breckels L.M., Gatto L., Christoforou A., Groen A.J., Kathryn Lilley
  K.S. and Trotter M.W.  \emph{The effect of organelle discovery upon
    sub-cellular protein localisation.}  J Proteomics,
  S1874-3919(13)00094-8 (2013)
\end{quote}

For an introduction to spatial proteomics data analysis:

\begin{quote}
  Gatto L, Breckels LM, Burger T, Nightingale DJ, Groen AJ, Campbell
  C, Nikolovski N, Mulvey CM, Christoforou A, Ferro M, Lilley
  KS. \emph{A foundation for reliable spatial proteomics data
    analysis}. Mol Cell Proteomics. 2014
  Aug;13(8):1937-52. doi:10.1074/mcp.M113.036350.
\end{quote}

The \Biocpkg{pRoloc} package contains additional vignettes and
reference material:

\begin{itemize}
\item \emph{pRoloc-tutorial}: pRoloc tutorial.
\item \emph{pRoloc-ml}: Machine learning techniques available in pRoloc.
\item \emph{pRoloc-transfer-learning}: A transfer learning algorithm
  for spatial proteomics.
\item \emph{pRoloc-goannotations}: Annotating spatial proteomics data.
\item \emph{HUPO\_2011\_poster}: HUPO 2011 poster: \textit{pRoloc -- A
    unifying bioinformatics framework for organelle proteomics}
\item \textit{HUPO\_2014\_poster}: HUPO 2014 poster: \textit{A
    state-of-the-art machine learning pipeline for the analysis of
    spatial proteomics data}
\end{itemize}